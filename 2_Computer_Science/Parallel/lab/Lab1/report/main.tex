%!TeX program = xelatex
\documentclass{SYSUReport}
\usepackage{tabularx}
\usepackage{array}
\usepackage{booktabs}
\usepackage{caption}
\usepackage{float}
\usepackage{listings}
\usepackage{xcolor}

\lstset{
    basicstyle=\ttfamily\small,
    keywordstyle=\color{blue},
    commentstyle=\color{green},
    stringstyle=\color{red},
    numbers=left,
    numberstyle=\tiny\color{gray},
    frame=single,
    breaklines=true,
    tabsize=4
}

\headl{}
\headc{}
\headr{并行程序设计与算法实验}
\lessonTitle{并行程序设计与算法实验}
\reportTitle{Lab1-基于MPI的并行矩阵乘法}
\stuname{林隽哲}
\stuid{21312450}
\inst{计算机学院}
\major{计算机科学与技术}
\date{2025年4月7日}

\begin{document}

\cover
\thispagestyle{empty}
\clearpage

\section{实验目的}
\begin{itemize}
   \item 掌握 MPI 程序的编译和运行方法。
    \item 理解 MPI 点对点通信的基本原理。
    \item 了解 MPI 程序的 GDB 调试流程。
\end{itemize}

\section{实验内容}
\begin{itemize}
   \item 使用 MPI 点对点通信实现并行矩阵乘法。
    \item 设置进程数量(1$\sim$16)及矩阵规模(128$\sim$2048)。
    \item 根据运行时间,分析程序的并行性能。
\end{itemize}


\section{实验结果}

\subsection{实验核心代码}
% \lstinputlisting[language=C]{../code/matrix_multiply.c}
\begin{lstlisting}[language=C]
int main(int argc, char *argv[]) {
    int n = 128;
    int rank, size;
    double *A = NULL, *B = NULL, *C = NULL;
    double *local_A = NULL, *local_C = NULL;
    double start_time, end_time;
    
    MPI_Init(&argc, &argv);
    MPI_Comm_rank(MPI_COMM_WORLD, &rank);
    MPI_Comm_size(MPI_COMM_WORLD, &size);
    
    if (n % size != 0) {
        if (rank == 0) {
            printf("Matrix size (%d) must be divisible by number of processes (%d)\n", n, size);
        }
        MPI_Finalize();
        return 1;
    }
    
    int rows_per_proc = n / size;
    
    // Allocate memory for local matrices
    local_A = (double*)malloc(rows_per_proc * n * sizeof(double));
    local_C = (double*)malloc(rows_per_proc * n * sizeof(double));
    memset(local_C, 0, rows_per_proc * n * sizeof(double));
    
    // Root process initializes matrices
    if (rank == 0) {
        A = (double*)malloc(n * n * sizeof(double));
        B = (double*)malloc(n * n * sizeof(double));
        C = (double*)malloc(n * n * sizeof(double));
        
        srand(time(NULL));
        initialize_matrix(A, n, n);
        initialize_matrix(B, n, n);
        
        if (n <= 10) {
            printf("Matrix A:\n");
            print_matrix(A, n, n);
            printf("Matrix B:\n");
            print_matrix(B, n, n);
        }
    }
    
    start_time = MPI_Wtime();
    
    // Distribute matrix A using point-to-point communication
    if (rank == 0) {
        // Root process sends parts of A to other processes
        for (int dest = 1; dest < size; dest++) {
            MPI_Send(A + dest * rows_per_proc * n, 
                    rows_per_proc * n, 
                    MPI_DOUBLE, 
                    dest, 
                    0, 
                    MPI_COMM_WORLD);
        }
        // Root process keeps its own part
        memcpy(local_A, A, rows_per_proc * n * sizeof(double));
    } else {
        // Other processes receive their part of A
        MPI_Recv(local_A, 
                rows_per_proc * n, 
                MPI_DOUBLE, 
                0, 
                0, 
                MPI_COMM_WORLD, 
                MPI_STATUS_IGNORE);
    }
    
    // Distribute matrix B using point-to-point communication
    if (rank == 0) {
        // Root process sends B to all other processes
        for (int dest = 1; dest < size; dest++) {
            MPI_Send(B, 
                    n * n, 
                    MPI_DOUBLE, 
                    dest, 
                    1, 
                    MPI_COMM_WORLD);
        }
    } else {
        // Other processes allocate memory for B and receive it
        B = (double*)malloc(n * n * sizeof(double));
        MPI_Recv(B, 
                n * n, 
                MPI_DOUBLE, 
                0, 
                1, 
                MPI_COMM_WORLD, 
                MPI_STATUS_IGNORE);
    }
    
    // Perform local matrix multiplication
    for (int i = 0; i < rows_per_proc; i++) {
        for (int j = 0; j < n; j++) {
            for (int k = 0; k < n; k++) {
                local_C[i * n + j] += local_A[i * n + k] * B[k * n + j];
            }
        }
    }
    
    // Gather results using point-to-point communication
    if (rank == 0) {
        // Root process receives results from other processes
        for (int src = 1; src < size; src++) {
            MPI_Recv(C + src * rows_per_proc * n, 
                    rows_per_proc * n, 
                    MPI_DOUBLE, 
                    src, 
                    2, 
                    MPI_COMM_WORLD, 
                    MPI_STATUS_IGNORE);
        }
        // Root process copies its own result
        memcpy(C, local_C, rows_per_proc * n * sizeof(double));
    } else {
        // Other processes send their results to root
        MPI_Send(local_C, 
                rows_per_proc * n, 
                MPI_DOUBLE, 
                0, 
                2, 
                MPI_COMM_WORLD);
    }
    
    end_time = MPI_Wtime();
    
    // Print result matrix if small enough
    if (rank == 0) {
        if (n <= 10) {
            printf("Result Matrix C:\n");
            print_matrix(C, n, n);
        }
        printf("Matrix size: %d x %d\n", n, n);
        printf("Number of processes: %d\n", size);
        printf("Execution time: %f seconds\n", end_time - start_time);
    }
    
    // Free memory
    if (rank == 0) {
        free(A);
        free(C);
    }
    free(B);
    free(local_A);
    free(local_C);
    
    MPI_Finalize();
    return 0;
}
\end{lstlisting}

\subsection{运行时间}
根据运行结果,填入下表以记录不同进程数和矩阵规模下的运行时间:
\begin{table}[H]
\centering
\begin{tabular}{|c|lllll|}
\hline
\multirow{2}{*}{进程数} & \multicolumn{5}{c|}{矩阵规模}                                                                        \\ \cline{2-6} 
 & \multicolumn{1}{l|}{128} & \multicolumn{1}{l|}{256} & \multicolumn{1}{l|}{512} & \multicolumn{1}{l|}{1024} & 2048 \\ \hline
1                    & \multicolumn{1}{l|}{0.008940} & \multicolumn{1}{l|}{0.075898} & \multicolumn{1}{l|}{0.705587} & \multicolumn{1}{l|}{6.545637} & 56.328150 \\ \hline
2                    & \multicolumn{1}{l|}{0.004838} & \multicolumn{1}{l|}{0.033641} & \multicolumn{1}{l|}{0.367404} & \multicolumn{1}{l|}{3.555477} & 29.233684 \\ \hline
4                    & \multicolumn{1}{l|}{0.008679} & \multicolumn{1}{l|}{0.049823} & \multicolumn{1}{l|}{0.292343} & \multicolumn{1}{l|}{2.215296} & 17.302856 \\ \hline
8                    & \multicolumn{1}{l|}{0.009288} & \multicolumn{1}{l|}{0.074531} & \multicolumn{1}{l|}{0.210484} & \multicolumn{1}{l|}{1.316330} & 11.103513 \\ \hline
16                   & \multicolumn{1}{l|}{0.067242} & \multicolumn{1}{l|}{0.024242} & \multicolumn{1}{l|}{0.161879} & \multicolumn{1}{l|}{1.270833} & 9.155628 \\ \hline
\end{tabular}
\caption{不同进程数和矩阵规模下的运行时间(单位:秒)}
\end{table}

根据上表的结果数据,我可以进行以下性能分析:

\subsection{加速比分析}
将进程数为1的情况作为基准,计算不同进程数下的加速比:

\begin{table}[H]
\centering
\begin{tabular}{|c|lllll|}
\hline
\multirow{2}{*}{进程数} & \multicolumn{5}{c|}{矩阵规模}                                                                        \\ \cline{2-6} 
 & \multicolumn{1}{l|}{128} & \multicolumn{1}{l|}{256} & \multicolumn{1}{l|}{512} & \multicolumn{1}{l|}{1024} & 2048 \\ \hline
1                    & \multicolumn{1}{l|}{1.00} & \multicolumn{1}{l|}{1.00} & \multicolumn{1}{l|}{1.00} & \multicolumn{1}{l|}{1.00} & 1.00 \\ \hline
2                    & \multicolumn{1}{l|}{1.85} & \multicolumn{1}{l|}{2.26} & \multicolumn{1}{l|}{1.92} & \multicolumn{1}{l|}{1.84} & 1.93 \\ \hline
4                    & \multicolumn{1}{l|}{1.03} & \multicolumn{1}{l|}{1.52} & \multicolumn{1}{l|}{2.41} & \multicolumn{1}{l|}{2.96} & 3.26 \\ \hline
8                    & \multicolumn{1}{l|}{0.96} & \multicolumn{1}{l|}{1.02} & \multicolumn{1}{l|}{3.35} & \multicolumn{1}{l|}{4.97} & 5.07 \\ \hline
16                   & \multicolumn{1}{l|}{0.13} & \multicolumn{1}{l|}{3.13} & \multicolumn{1}{l|}{4.36} & \multicolumn{1}{l|}{5.15} & 6.15 \\ \hline
\end{tabular}
\caption{不同进程数和矩阵规模下的加速比}
\end{table}

\subsection{性能分析}

\subsubsection{小规模矩阵(128~256)}
\begin{itemize}
    \item 观察小规模矩阵下的运行时间,可以发现,随着进程数的增加,运行时间减少直到进程数为16。
    \item 观察小规模矩阵下的加速比,可以发现,随着进程数的增加,加速比先增加后减少。
    \item 进程数的增加将会同时导致通信开销的增加,当通信开销大于计算收益时,运行时间将会不降反增,加速比也将会降低。
\end{itemize}

\subsubsection{中大规模矩阵(512~2048)}
\begin{itemize}
    \item 观察中大规模矩阵的运行时间与加速比,可以发现,随着矩阵规模的增加,进程数的增加所带来的收益将会愈发明显。
\end{itemize}

\section{讨论题}
\begin{itemize}
    \item 在内存受限情况下,如何进行大规模矩阵乘法计算?

    \begin{itemize}
        \item 分块计算:将大矩阵分成多个小块,每次只加载部分数据到内存中进行计算。这种方法可以显著减少内存使用量,但会增加I/O操作。
        \item 分布式计算:将矩阵分布到多个计算节点上进行计算,每个节点只处理部分数据。
        \item 数据压缩:对矩阵数据进行压缩存储,在计算时再解压。
    \end{itemize}

    
    \item 如何提高大规模稀疏矩阵乘法性能?

    \begin{itemize}
        \item 使用稀疏矩阵存储格式:如CSR(Compressed Sparse Row)或CSC(Compressed Sparse Column)等压缩存储格式,只存储非零元素,减少内存使用和计算量。
        \item 对数据预处理:在计算前对矩阵进行预处理,如矩阵重排序,将非零元素尽可能聚集在一起,减少计算量。
        \item 并行计算:使用并行计算技术,如MPI,将矩阵分布到多个计算节点上进行计算,每个节点只处理部分数据。
    \end{itemize}

\end{itemize}
\end{document}