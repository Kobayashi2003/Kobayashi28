%!TeX program = xelatex
\documentclass{SYSUReport}
\usepackage{tabularx} % 在导言区添加此行
\usepackage{array}
\usepackage{booktabs}
\usepackage{caption}
% 根据个人情况修改
\headl{}
\headc{}
\headr{并行程序设计与算法实验}
\lessonTitle{并行程序设计与算法实验}
\reportTitle{Lab1-基于MPI的并行矩阵乘法}
\stuname{xxx}
\stuid{xxx}
\inst{计算机学院}
\major{xxx}
\date{2025年3月26日}

\begin{document}

% =============================================
% Part 1: 封面
% =============================================
\cover
\thispagestyle{empty} % 首页不显示页码
\clearpage

% % =============================================
% % Part 4: 正文内容
% % =============================================
% % 重置页码,并使用阿拉伯数字
% \pagenumbering{arabic}
% \setcounter{page}{1}

%%可选择这里也放一个标题
%\begin{center}
%    \title{ \Huge \textbf{{标题}}}
%\end{center}

\section{实验目的}
\begin{itemize}
   \item 掌握 MPI 程序的编译和运行方法。
    \item 理解 MPI 点对点通信的基本原理。
    \item 了解 MPI 程序的 GDB 调试流程。
\end{itemize}

\section{实验内容}
\begin{itemize}
   \item 使用 MPI 点对点通信实现并行矩阵乘法。
    \item 设置进程数量(1$\sim$16)及矩阵规模(128$\sim$2048)。
    \item 根据运行时间,分析程序的并行性能。
\end{itemize}


\section{实验结果}
\subsection{性能分析}
根据运行结果,填入下表以记录不同进程数和矩阵规模下的运行时间:
% Please add the following required packages to your document preamble:
% \usepackage{booktabs}
% \usepackage{multirow}
% \usepackage{graphicx}
% Please add the following required packages to your document preamble:
% \usepackage{booktabs}
% \usepackage{multirow}
% \usepackage{graphicx}
% Please add the following required packages to your document preamble:
% \usepackage{multirow}
% \usepackage{graphicx}
% Please add the following required packages to your document preamble:
% \usepackage{multirow}
\begin{table}[htpb]
\centering
\begin{tabular}{|c|lllll|}
\hline
\multirow{2}{*}{进程数} & \multicolumn{5}{c|}{矩阵规模}                                                                        \\ \cline{2-6} 
 & \multicolumn{1}{l|}{128} & \multicolumn{1}{l|}{256} & \multicolumn{1}{l|}{512} & \multicolumn{1}{l|}{1024} & 2048 \\ \hline
1                    & \multicolumn{1}{l|}{result} & \multicolumn{1}{l|}{result} & \multicolumn{1}{l|}{result} & \multicolumn{1}{l|}{result} &  result\\ \hline
2                    & \multicolumn{1}{l|}{result} & \multicolumn{1}{l|}{result} & \multicolumn{1}{l|}{result} & \multicolumn{1}{l|}{result} &  result\\ \hline
4                    & \multicolumn{1}{l|}{result} & \multicolumn{1}{l|}{result} & \multicolumn{1}{l|}{result} & \multicolumn{1}{l|}{result} &result  \\ \hline
9                    & \multicolumn{1}{l|}{result} & \multicolumn{1}{l|}{result} & \multicolumn{1}{l|}{result} & \multicolumn{1}{l|}{result} &result  \\ \hline
16                   & \multicolumn{1}{l|}{result} & \multicolumn{1}{l|}{result} & \multicolumn{1}{l|}{result} & \multicolumn{1}{l|}{result} &result  \\ \hline
\end{tabular}
\end{table}
\section{讨论题}
\begin{itemize}
    \item 在内存受限情况下,如何进行大规模矩阵乘法计算?

    回答:......
    \item 如何提高大规模稀疏矩阵乘法性能?

    回答:......
\end{itemize}
注:实验代码以zip格式另提交;最终提交内容包括实验报告(pdf格式)和实验代码(zip压缩包格式)
\end{document}