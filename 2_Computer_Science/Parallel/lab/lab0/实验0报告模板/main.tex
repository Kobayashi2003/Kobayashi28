%!TeX program = xelatex
\documentclass{SYSUReport}
\usepackage{tabularx} % 在导言区添加此行

% 根据个人情况修改
\headl{}
\headc{}
\headr{并行程序设计与算法实验}
\lessonTitle{并行程序设计与算法实验}
\reportTitle{Lab0-环境设置与串行矩阵乘法}
\stuname{林隽哲}
\stuid{21312450}
\inst{计算机学院}
\major{计算机科学与技术}
\date{\today}

\begin{document}

% =============================================
% Part 1: 封面
% =============================================
\cover
\thispagestyle{empty} % 首页不显示页码
\clearpage

% % =============================================
% % Part 4: 正文内容
% % =============================================
% % 重置页码,并使用阿拉伯数字
% \pagenumbering{arabic}
% \setcounter{page}{1}

%%可选择这里也放一个标题
%\begin{center}
%    \title{ \Huge \textbf{{标题}}}
%\end{center}

\section{实验目的}
\begin{itemize}
   \item 理解并行程序设计的基本概念与理论。
    \item 掌握使用并行编程模型实现常见算法的能力。
    \item 学习评估并行程序性能的指标及其优化方法。
\end{itemize}

\section{实验内容}
\begin{itemize}
    \item 设计并实现以下矩阵乘法版本:
    \begin{itemize}
        \item 使用C/C++语言实现一个串行矩阵乘法。
        \item 比较不同编译选项、实现方式、算法或库对性能的影响:
        \begin{itemize}
            \item 使用Python实现的矩阵乘法。
            \item 使用C/C++实现的基本矩阵乘法。
            \item 调整循环顺序优化矩阵乘法。
            \item 应用编译优化提高性能。
            \item 使用循环展开技术优化矩阵乘法。
            \item 使用Intel MKL库进行矩阵乘法运算。
        \end{itemize}
    \end{itemize}
    \item 生成随机矩阵A和B,进行矩阵乘法运算得到矩阵C。
    \item 衡量各版本的运行时间、加速比、浮点性能等。
    \item 分析不同实现版本对性能的影响。
\end{itemize}


\section{实验结果}
\begin{table}[h]
    \centering
    \begin{tabular}{|c|c|c|p{1.5cm}|p{1.5cm}|c|p{2cm}|}
        \hline
        版本 & 实现描述 & 运行时间& 相对加速比 & 绝对加速比 & 浮点性能 & 峰值性能百分比 \\
        \hline
        1 & Python &  &  &  &  &\\ 
        \hline
        2 & C/C++ &  &  &  & & \\ 
        \hline
        3 & 调整循环顺序 &  &  &  & & \\ 
        \hline
        4 & 编译优化 &  &  &  & & \\ 
        \hline
        5 & 循环展开 &  &  &  &  &\\ 
        \hline
        6 & Intel MKL &  &  &  & & \\ 
        \hline
    \end{tabular}
\end{table}
\section{实验分析}
\begin{itemize}
    \item ......
    \item ......
    \item ......
\end{itemize}
注:实验代码以zip格式另提交;最终提交内容包括实验报告(pdf格式)和实验代码(zip压缩包格式)
\end{document}