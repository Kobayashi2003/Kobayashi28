%!TeX program = xelatex
\documentclass{SYSUReport}
\usepackage{tabularx} % 在导言区添加此行
\usepackage{float}
% 根据个人情况修改
\headl{}
\headc{}
\headr{并行程序设计与算法实验}
\lessonTitle{并行程序设计与算法实验}
\reportTitle{Lab3-Pthreads并行矩阵乘法与数组求和}
\stuname{xxx}
\stuid{xxx}
\inst{计算机学院}
\major{xxx}
\date{2025年4月9日}

\begin{document}

% =============================================
% Part 1: 封面
% =============================================
\cover
\thispagestyle{empty} % 首页不显示页码
\clearpage

% % =============================================
% % Part 4: 正文内容
% % =============================================
% % 重置页码,并使用阿拉伯数字
% \pagenumbering{arabic}
% \setcounter{page}{1}

%%可选择这里也放一个标题
%\begin{center}
%    \title{ \Huge \textbf{{标题}}}
%\end{center}

\section{实验目的}
\begin{itemize}
    \item Pthreads程序编写、运行与调试
    \item 多线程并行矩阵乘法
    \item 多线程并行数组求和
\end{itemize}

\section{实验内容}
\begin{itemize}
    \item 掌握Pthreads编程的基本流程
    \item 理解线程间通信与资源共享机制
    \item 通过性能分析明确线程数、数据规模与加速比的关系
\end{itemize}
\subsection{并行矩阵乘法}
\begin{itemize}
    \item 使用Pthreads实现并行矩阵乘法
    \item 随机生成$m \times n$的矩阵$A$及$n \times k$的矩阵$B$
    \item 通过多线程计算矩阵乘积$C = A \times B$
    \item 调整线程数量(1-16)和矩阵规模(128-2048),记录计算时间
    \item 分析并行性能(时间、效率、可扩展性)
    \item 选做:分析不同数据划分方式的影响
     \begin{itemize}
        \item 请描述你的数据/任务划分方式。
        \item 回答:......
    \end{itemize}
\end{itemize}
\subsection{并行数组求和}
\begin{itemize}
    \item 使用Pthreads实现并行数组求和
    \item 随机生成长度为$n$的整型数组$A$,$n$取值范围[1M, 128M]
    \item 通过多线程计算数组元素和$s = \sum_{i=1}^{n}A_i$
    \item 调整线程数量(1-16)和数组规模(1M-128M),记录计算时间
    \item 分析并行性能(时间、效率、可扩展性)
    \item 选做:分析不同聚合方式的影响
     \begin{itemize}
        \item 请描述你的聚合方式。
        \item 回答:......
    \end{itemize}
\end{itemize}
\section{实验结果}
\subsection{并行矩阵乘法}

    \begin{table}[H]
\centering
\caption{并行矩阵乘法在不同线程数下的运行时间}
\begin{tabular}{cccccc}
\toprule
矩阵规模 & 1线程 & 2线程 & 4线程 & 8线程 & 16线程 \\
\midrule
128×128 & & & & & \\
256×256 & & & & & \\
512×512 & & & & & \\
1024×1024 & & & & & \\
2048×2048 & & & & & \\
\bottomrule
\end{tabular}
\end{table}
\subsection{并行数组求和}
\begin{table}[h]
\centering
\caption{数组求和不同线程数下的运行时间}
\begin{tabular}{cccccc}
\toprule
数组规模 & 1线程 & 2线程 & 4线程 & 8线程 & 16线程 \\
\midrule
1M & & & & & \\
4M & & & & & \\
16M & & & & & \\
64M & & & & & \\
128M & & & & & \\
\bottomrule
\end{tabular}
\end{table}




\section{实验分析}
\subsection{并行矩阵乘法}
\begin{itemize}
    \item 线程数量对性能的影响分析:
    \item 矩阵规模对并行效率的影响:
    \item 可扩展性分析:
    \item (选做)不同数据划分方式的比较:
\end{itemize}

\subsection{并行数组求和}
\begin{itemize}
    \item 线程数量对性能的影响分析:
    \item 数组规模对并行效率的影响:
    \item 可扩展性分析:
    \item (选做)不同聚合方式的比较:
\end{itemize}
注:实验代码以zip格式另提交;最终提交内容包括实验报告(pdf格式)和实验代码(zip压缩包格式)
\end{document}