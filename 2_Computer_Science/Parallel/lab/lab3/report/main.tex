%!TeX program = xelatex
\documentclass{SYSUReport}
\usepackage{tabularx}
\usepackage{float}

\headl{}
\headc{}
\headr{并行程序设计与算法实验}
\lessonTitle{并行程序设计与算法实验}
\reportTitle{Lab3-Pthreads并行矩阵乘法与数组求和}
\stuname{林隽哲}
\stuid{21312450}
\inst{计算机学院}
\major{计算机科学与技术}
\date{2025年4月9日}

\begin{document}
\cover
\thispagestyle{empty}
\clearpage

\section{实验目的}
\begin{itemize}
    \item Pthreads程序编写、运行与调试
    \item 多线程并行矩阵乘法
    \item 多线程并行数组求和
\end{itemize}

\section{实验内容}
\begin{itemize}
    \item 掌握Pthreads编程的基本流程
    \item 理解线程间通信与资源共享机制
    \item 通过性能分析明确线程数、数据规模与加速比的关系
\end{itemize}
\subsection{并行矩阵乘法}
\begin{itemize}
    \item 使用Pthreads实现并行矩阵乘法
    \item 随机生成$m \times n$的矩阵$A$及$n \times k$的矩阵$B$
    \item 通过多线程计算矩阵乘积$C = A \times B$
    \item 调整线程数量(1-16)和矩阵规模(128-2048),记录计算时间
    \item 分析并行性能(时间、效率、可扩展性)
\end{itemize}
\subsection{并行数组求和}
\begin{itemize}
    \item 使用Pthreads实现并行数组求和
    \item 随机生成长度为$n$的整型数组$A$,$n$取值范围[1M, 128M]
    \item 通过多线程计算数组元素和$s = \sum_{i=1}^{n}A_i$
    \item 调整线程数量(1-16)和数组规模(1M-128M),记录计算时间
    \item 分析并行性能(时间、效率、可扩展性)
\end{itemize}
\section{实验结果}
\subsection{并行矩阵乘法}

\begin{table}[H]
\centering
\caption{并行矩阵乘法在不同线程数下的运行时间}
\begin{tabular}{cccccc}
\toprule
矩阵规模 & 1线程 & 2线程 & 4线程 & 8线程 & 16线程 \\
\midrule
128×128 & 0.010 & 0.006 & 0.007 & 0.011 & 0.010 \\
256×256 & 0.058 & 0.032 & 0.026 & 0.021 & 0.024 \\
512×512 & 0.567 & 0.281 & 0.168 & 0.119 & 0.124 \\
1024×1024 & 5.671 & 2.902 & 1.602 & 1.011 & 0.951 \\
2048×2048 & 82.575 & 43.418 & 21.989 & 14.054 & 13.390 \\
\bottomrule
\end{tabular}
\end{table}
\subsection{并行数组求和}
\begin{table}[h]
\centering
\caption{数组求和不同线程数下的运行时间}
\begin{tabular}{cccccc}
\toprule
数组规模 & 1线程 & 2线程 & 4线程 & 8线程 & 16线程 \\
\midrule
1M & 0.002 & 0.002 & 0.005 & 0.005 & 0.011 \\
4M & 0.009 & 0.025 & 0.010 & 0.013 & 0.013 \\
16M & 0.033 & 0.044 & 0.058 & 0.032 & 0.034 \\
64M & 0.149 & 0.322 & 0.203 & 0.109 & 0.103 \\
128M & 0.255 & 0.320 & 0.456 & 0.326 & 0.151 \\
\bottomrule
\end{tabular}
\end{table}

\section{实验分析}
\subsection{并行矩阵乘法}
\begin{itemize}
    \item 线程数量对性能的影响分析:
    \begin{itemize}
        \item 对于小规模的矩阵,多线程带来的性能提升不明显,甚至有时会出现性能下降,这是因为线程创建和管理的开销超过了并行计算带来的收益。
        \item 对于大规模的矩阵,多线程的优势更加明显,可以从2048×2048的矩阵乘法的结果中看出,单线程需要82.575秒,而使用16个线程时仅需13.390秒,加速比约为6.17。
    \end{itemize}
    
    \item 矩阵规模对并行效率的影响:
    \begin{itemize}
        \item 当矩阵规模较小时,线程间的通信开销和线程管理开销占比较大,导致并行效率较低。
        \item 矩阵规模越大,并行效率越高。对于128×128的矩阵,16线程的加速比仅为1.0,而对于2048×2048的矩阵,加速比达到6.17。
    \end{itemize}
    
    \item 可扩展性分析:
    \begin{itemize}
        \item 从2线程到16线程,性能提升呈现递减趋势,说明系统存在一定的可扩展性限制。
        \item 这种限制可能来自多个方面:内存带宽限制、CPU缓存竞争、线程调度开销以及矩阵规模等。
        \item 对于2048×2048的矩阵,从8线程增加到16线程时,性能提升仅为5\%,说明对于2048×2048的矩阵,16线程已经接近性能瓶颈。如果继续增加线程数,线程调度开销将超过并行计算带来的收益。
    \end{itemize}
\end{itemize}

\subsection{并行数组求和}
\begin{itemize}
    \item 线程数量对性能的影响分析:
    \begin{itemize}
        \item 对于小规模数组(1M),多线程反而导致性能下降,这是因为线程创建和同步的开销超过了并行计算带来的收益。
        \item 对于大规模数组(128M),16线程相比单线程有约1.69倍的加速比,但性能提升不如矩阵乘法明显。
        \item 在某些情况下(如64M数组),2线程的性能反而比4线程差,这可能是因为线程调度和内存访问模式的影响。
    \end{itemize}
    
    \item 数组规模对并行效率的影响:
    \begin{itemize}
        \item 数组规模越大,并行效率越高。对于1M的数组,多线程几乎没有性能提升,而对于128M的数组,16线程有显著加速。
        \item 当数组规模较小时,线程间的通信开销和线程管理开销占比较大,导致并行效率较低。
        \item 大规模数组可以更好地分摊线程创建和同步的开销,使得并行计算更加高效。
    \end{itemize}
    
    \item 可扩展性分析:
    \begin{itemize}
        \item 数组求和的并行效率明显低于矩阵乘法,这是因为数组求和的计算密度较低,内存访问模式更简单。
        \item 从2线程到16线程,性能提升呈现不稳定的趋势,说明其可扩展性较差。
        \item 这种限制主要来自内存带宽和缓存竞争,因为数组求和主要是内存密集型操作。
    \end{itemize}
\end{itemize}
\end{document}