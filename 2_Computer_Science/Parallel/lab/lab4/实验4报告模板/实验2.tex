%!TeX program = xelatex
\documentclass{SYSUReport}
\usepackage{tabularx} % 在导言区添加此行
\usepackage{float}
% 根据个人情况修改
\headl{}
\headc{}
\headr{并行程序设计与算法实验}
\lessonTitle{并行程序设计与算法实验}
\reportTitle{Lab2-基于MPI的并行矩阵乘法(进阶)}
\stuname{xxx}
\stuid{xxx}
\inst{计算机学院}
\major{xxx}
\date{2025年4月2日}

\begin{document}

% =============================================
% Part 1: 封面
% =============================================
\cover
\thispagestyle{empty} % 首页不显示页码
\clearpage

% % =============================================
% % Part 4: 正文内容
% % =============================================
% % 重置页码,并使用阿拉伯数字
% \pagenumbering{arabic}
% \setcounter{page}{1}

%%可选择这里也放一个标题
%\begin{center}
%    \title{ \Huge \textbf{{标题}}}
%\end{center}

\section{实验目的}
\begin{itemize}
    \item 掌握MPI集合通信在并行矩阵乘法中的应用
    \item 学习使用MPI\_Type\_create\_struct创建派生数据类型
    \item 分析不同通信方式和任务划分对并行性能的影响
    \item 研究并行程序的扩展性和性能优化方法
\end{itemize}

\section{实验内容}
\begin{itemize}
    \item 使用MPI集合通信实现并行矩阵乘法
    \item 使用MPI\_Type\_create\_struct聚合进程内变量后通信
    \item 选做:尝试不同数据/任务划分方式
    \begin{itemize}
        \item 请描述你的数据/任务划分方式。
        \item 回答:......
    \end{itemize}
\end{itemize}


\section{实验结果}
\subsection{性能分析}
根据运行结果,填入下表以记录不同线程数量和矩阵规模下的运行时间:
\begin{table}[H]
\centering
\caption{用MPI集合通信实现}
\label{表1}
\begin{tabular}{|c|lllll|}
\hline
\multirow{2}{*}{进程数} & \multicolumn{5}{c|}{矩阵规模}                                                                        \\ \cline{2-6} 
 & \multicolumn{1}{l|}{128} & \multicolumn{1}{l|}{256} & \multicolumn{1}{l|}{512} & \multicolumn{1}{l|}{1024} & 2048 \\ \hline
1                    & \multicolumn{1}{l|}{} & \multicolumn{1}{l|}{} & \multicolumn{1}{l|}{} & \multicolumn{1}{l|}{} &  \\ \hline
2                    & \multicolumn{1}{l|}{} & \multicolumn{1}{l|}{} & \multicolumn{1}{l|}{} & \multicolumn{1}{l|}{} &  \\ \hline
4                    & \multicolumn{1}{l|}{} & \multicolumn{1}{l|}{} & \multicolumn{1}{l|}{} & \multicolumn{1}{l|}{} &  \\ \hline
8                    & \multicolumn{1}{l|}{} & \multicolumn{1}{l|}{} & \multicolumn{1}{l|}{} & \multicolumn{1}{l|}{} &  \\ \hline
16                   & \multicolumn{1}{l|}{} & \multicolumn{1}{l|}{} & \multicolumn{1}{l|}{} & \multicolumn{1}{l|}{} &  \\ \hline
\end{tabular}
\end{table}

\begin{table}[H]
\centering
\caption{用MPI\_Type\_create\_struct聚合进程内变量后通信}
\label{表2}
\begin{tabular}{|c|lllll|}
\hline
\multirow{2}{*}{进程数} & \multicolumn{5}{c|}{矩阵规模}                                                                        \\ \cline{2-6} 
 & \multicolumn{1}{l|}{128} & \multicolumn{1}{l|}{256} & \multicolumn{1}{l|}{512} & \multicolumn{1}{l|}{1024} & 2048 \\ \hline
1                    & \multicolumn{1}{l|}{} & \multicolumn{1}{l|}{} & \multicolumn{1}{l|}{} & \multicolumn{1}{l|}{} &  \\ \hline
2                    & \multicolumn{1}{l|}{} & \multicolumn{1}{l|}{} & \multicolumn{1}{l|}{} & \multicolumn{1}{l|}{} &  \\ \hline
4                    & \multicolumn{1}{l|}{} & \multicolumn{1}{l|}{} & \multicolumn{1}{l|}{} & \multicolumn{1}{l|}{} &  \\ \hline
8                   & \multicolumn{1}{l|}{} & \multicolumn{1}{l|}{} & \multicolumn{1}{l|}{} & \multicolumn{1}{l|}{} &  \\ \hline
16                   & \multicolumn{1}{l|}{} & \multicolumn{1}{l|}{} & \multicolumn{1}{l|}{} & \multicolumn{1}{l|}{} &  \\ \hline
\end{tabular}
\end{table}

\begin{table}[H]
\centering
\caption{选做题实验结果请填写在此表}
\label{表3}
\begin{tabular}{|c|lllll|}
\hline
\multirow{2}{*}{进程数} & \multicolumn{5}{c|}{矩阵规模}                                                                        \\ \cline{2-6} 
 & \multicolumn{1}{l|}{128} & \multicolumn{1}{l|}{256} & \multicolumn{1}{l|}{512} & \multicolumn{1}{l|}{1024} & 2048 \\ \hline
1                    & \multicolumn{1}{l|}{} & \multicolumn{1}{l|}{} & \multicolumn{1}{l|}{} & \multicolumn{1}{l|}{} &  \\ \hline
2                    & \multicolumn{1}{l|}{} & \multicolumn{1}{l|}{} & \multicolumn{1}{l|}{} & \multicolumn{1}{l|}{} &  \\ \hline
4                    & \multicolumn{1}{l|}{} & \multicolumn{1}{l|}{} & \multicolumn{1}{l|}{} & \multicolumn{1}{l|}{} &  \\ \hline
8                    & \multicolumn{1}{l|}{} & \multicolumn{1}{l|}{} & \multicolumn{1}{l|}{} & \multicolumn{1}{l|}{} &  \\ \hline
16                   & \multicolumn{1}{l|}{} & \multicolumn{1}{l|}{} & \multicolumn{1}{l|}{} & \multicolumn{1}{l|}{} &  \\ \hline
\end{tabular}
\end{table}


\section{实验分析}
根据运行时间,分析程序并行性能及扩展性
\begin{itemize}
    \item 使用MPI集合通信实现并行矩阵乘法:
    \item 使用MPI\_Type\_create\_struct聚合进程内变量后通信:
    \item 你的方法(选做):
\end{itemize}
注:实验代码以zip格式另提交;最终提交内容包括实验报告(pdf格式)和实验代码(zip压缩包格式)
\end{document}