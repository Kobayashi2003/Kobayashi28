%!TeX program = xelatex
\documentclass{SYSUReport}
\usepackage{tabularx} % 在导言区添加此行
\usepackage{float}
% 根据个人情况修改
\headl{}
\headc{}
\headr{并行程序设计与算法实验}
\lessonTitle{并行程序设计与算法实验}
\reportTitle{Lab4-Pthreads并行方程求解与蒙特卡洛}
\stuname{xxx}
\stuid{xxx}
\inst{计算机学院}
\major{xxx}
\date{2025年4月16日}

\begin{document}

% =============================================
% Part 1: 封面
% =============================================
\cover
\thispagestyle{empty} % 首页不显示页码
\clearpage

% % =============================================
% % Part 4: 正文内容
% % =============================================
% % 重置页码,并使用阿拉伯数字
% \pagenumbering{arabic}
% \setcounter{page}{1}

%%可选择这里也放一个标题
%\begin{center}
%    \title{ \Huge \textbf{{标题}}}
%\end{center}

\section{实验目的}
\begin{itemize}
    \item 深入理解Pthreads同步机制:条件变量、互斥锁
    \item 评估多线程加速性能
\end{itemize}

\section{实验内容}
\begin{itemize}
    \item 多线程一元二次方程求解
    \item 多线程圆周率计算
\end{itemize}
\subsection{方程求解}
\begin{itemize}
   \item 使用Pthread多线程及求根公式实现并行一元二次方程求解。
    \item 方程形式为 $x = \frac{-b\pm\sqrt{b^2-4ac}}{2a}$,其中判别式的计算、求根公式的中间值分别由不同的线程完成。
    \item 通过条件变量识别何时线程完成了所需计算,并分析计算任务依赖关系及程序并行性能。
    \end{itemize}
\subsection{蒙特卡洛求圆周率$pi$的近似值}
\begin{itemize}
   \item 使用Pthread创建多线程,并行生成正方形内的$n$个随机点。
    \item 统计落在正方形内切圆内点数,估算$\pi$的值。
    \item 设置线程数量(1-16)及随机点数量(1024-65536),观察对近似精度及程序并行性能的影响。
\end{itemize}
\section{实验结果与分析}
\subsection{方程求解}
\begin{itemize}
    \item 分析不同线程配置下的求解时间,评估并行化带来的性能提升。
    \item 对比单线程与多线程方案在处理相同方程时的表现,讨论可能存在的瓶颈或优化空间。
\end{itemize}
   
\subsection{蒙特卡洛方法求圆周率}
\begin{itemize}
\item 比较不同线程数量和随机点数量下圆周率估计的准确性和计算速度。
    \item 讨论增加线程数量是否总能提高计算效率,以及其对圆周率估计精度的影响。
    \item 提供实验数据图表,展示随着线程数和随机点数的变化,计算效率和精度的趋势。
    \item 分析实验过程中遇到的问题,如同步问题、负载不均等,并提出相应的解决策略。
\end{itemize}
注:实验报告格式参考本模板,可在此基础上进行修改;实验代码以zip格式另提交;最终提交内容包括实验报告(pdf格式)和实验代码(zip压缩包格式)
\end{document}