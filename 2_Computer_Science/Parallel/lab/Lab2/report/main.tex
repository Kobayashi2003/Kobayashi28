%!TeX program = xelatex
\documentclass{SYSUReport}
\usepackage{tabularx} 
\usepackage{float}

\headl{}
\headc{}
\headr{并行程序设计与算法实验}
\lessonTitle{并行程序设计与算法实验}
\reportTitle{Lab2-基于MPI的并行矩阵乘法(进阶)}
\stuname{林隽哲}
\stuid{21312450}
\inst{计算机学院}
\major{计算机科学与技术}
\date{2025年4月2日}

\begin{document}

\cover
\thispagestyle{empty} 
\clearpage

\section{实验目的}
\begin{itemize}
    \item 掌握MPI集合通信在并行矩阵乘法中的应用
    \item 学习使用MPI\_Type\_create\_struct创建派生数据类型
    \item 分析不同通信方式和任务划分对并行性能的影响
    \item 研究并行程序的扩展性和性能优化方法
\end{itemize}

\section{实验内容}
\begin{itemize}
    \item 使用MPI集合通信实现并行矩阵乘法
    \item 使用MPI\_Type\_create\_struct聚合进程内变量后通信
\end{itemize}


\section{实验结果}
\subsection{性能分析}
根据运行结果,填入下表以记录不同线程数量和矩阵规模下的运行时间:
\begin{table}[H]
\centering
\caption{用MPI集合通信实现}
\label{表1}
\begin{tabular}{|c|lllll|}
\hline
\multirow{2}{*}{进程数} & \multicolumn{5}{c|}{矩阵规模}                                                                        \\ \cline{2-6} 
 & \multicolumn{1}{l|}{128} & \multicolumn{1}{l|}{256} & \multicolumn{1}{l|}{512} & \multicolumn{1}{l|}{1024} & 2048 \\ \hline
1                    & \multicolumn{1}{l|}{0.009239} & \multicolumn{1}{l|}{0.072149} & \multicolumn{1}{l|}{0.670923} & \multicolumn{1}{l|}{6.924118} & 54.375005 \\ \hline
2                    & \multicolumn{1}{l|}{0.004372} & \multicolumn{1}{l|}{0.044670} & \multicolumn{1}{l|}{0.344490} & \multicolumn{1}{l|}{3.466872} & 30.935780 \\ \hline
4                    & \multicolumn{1}{l|}{0.002561} & \multicolumn{1}{l|}{0.029424} & \multicolumn{1}{l|}{0.216383} & \multicolumn{1}{l|}{2.023358} & 18.362430 \\ \hline
8                    & \multicolumn{1}{l|}{0.034996} & \multicolumn{1}{l|}{0.019056} & \multicolumn{1}{l|}{0.190298} & \multicolumn{1}{l|}{1.327365} & 11.270439 \\ \hline
16                   & \multicolumn{1}{l|}{0.050508} & \multicolumn{1}{l|}{0.010465} & \multicolumn{1}{l|}{0.131064} & \multicolumn{1}{l|}{1.200326} & 9.370801 \\ \hline
\end{tabular}
\end{table}

\begin{table}[H]
\centering
\caption{用MPI\_Type\_create\_struct聚合进程内变量后通信}
\label{表2}
\begin{tabular}{|c|lllll|}
\hline
\multirow{2}{*}{进程数} & \multicolumn{5}{c|}{矩阵规模}                                                                        \\ \cline{2-6} 
 & \multicolumn{1}{l|}{128} & \multicolumn{1}{l|}{256} & \multicolumn{1}{l|}{512} & \multicolumn{1}{l|}{1024} & 2048 \\ \hline
1                    & \multicolumn{1}{l|}{} & \multicolumn{1}{l|}{} & \multicolumn{1}{l|}{} & \multicolumn{1}{l|}{} &  \\ \hline
2                    & \multicolumn{1}{l|}{} & \multicolumn{1}{l|}{} & \multicolumn{1}{l|}{} & \multicolumn{1}{l|}{} &  \\ \hline
4                    & \multicolumn{1}{l|}{} & \multicolumn{1}{l|}{} & \multicolumn{1}{l|}{} & \multicolumn{1}{l|}{} &  \\ \hline
9                    & \multicolumn{1}{l|}{} & \multicolumn{1}{l|}{} & \multicolumn{1}{l|}{} & \multicolumn{1}{l|}{} &  \\ \hline
16                   & \multicolumn{1}{l|}{} & \multicolumn{1}{l|}{} & \multicolumn{1}{l|}{} & \multicolumn{1}{l|}{} &  \\ \hline
\end{tabular}
\end{table}


\section{实验分析}
根据运行时间,分析程序并行性能及扩展性
\begin{itemize}
    \item 使用MPI集合通信实现并行矩阵乘法:




    \item 使用MPI\_Type\_create\_struct聚合进程内变量后通信:





\end{itemize}
\end{document}