%!TeX program = xelatex
\documentclass{SYSUReport}
\usepackage{tabularx} 
\usepackage{float}

\headl{}
\headc{}
\headr{并行程序设计与算法实验}
\lessonTitle{并行程序设计与算法实验}
\reportTitle{Lab2-基于MPI的并行矩阵乘法(进阶)}
\stuname{林隽哲}
\stuid{21312450}
\inst{计算机学院}
\major{计算机科学与技术}
\date{2025年4月2日}

\begin{document}

\cover
\thispagestyle{empty} 
\clearpage

\section{实验目的}
\begin{itemize}
    \item 掌握MPI集合通信在并行矩阵乘法中的应用
    \item 学习使用MPI\_Type\_create\_struct创建派生数据类型
    \item 分析不同通信方式和任务划分对并行性能的影响
    \item 研究并行程序的扩展性和性能优化方法
\end{itemize}

\section{实验内容}
\begin{itemize}
    \item 使用MPI集合通信实现并行矩阵乘法
    \item 使用MPI\_Type\_create\_struct聚合进程内变量后通信
\end{itemize}


\section{实验结果}
\subsection{性能分析}
根据运行结果,填入下表以记录不同线程数量和矩阵规模下的运行时间:
\begin{table}[H]
\centering
\caption{用MPI集合通信实现}
\label{表1}
\begin{tabular}{|c|lllll|}
\hline
\multirow{2}{*}{进程数} & \multicolumn{5}{c|}{矩阵规模}                                                                        \\ \cline{2-6} 
 & \multicolumn{1}{l|}{128} & \multicolumn{1}{l|}{256} & \multicolumn{1}{l|}{512} & \multicolumn{1}{l|}{1024} & 2048 \\ \hline
1                    & \multicolumn{1}{l|}{0.009239} & \multicolumn{1}{l|}{0.072149} & \multicolumn{1}{l|}{0.670923} & \multicolumn{1}{l|}{6.924118} & 54.375005 \\ \hline
2                    & \multicolumn{1}{l|}{0.004372} & \multicolumn{1}{l|}{0.044670} & \multicolumn{1}{l|}{0.344490} & \multicolumn{1}{l|}{3.466872} & 30.935780 \\ \hline
4                    & \multicolumn{1}{l|}{0.002561} & \multicolumn{1}{l|}{0.029424} & \multicolumn{1}{l|}{0.216383} & \multicolumn{1}{l|}{2.023358} & 18.362430 \\ \hline
8                    & \multicolumn{1}{l|}{0.034996} & \multicolumn{1}{l|}{0.019056} & \multicolumn{1}{l|}{0.190298} & \multicolumn{1}{l|}{1.327365} & 11.270439 \\ \hline
16                   & \multicolumn{1}{l|}{0.050508} & \multicolumn{1}{l|}{0.010465} & \multicolumn{1}{l|}{0.131064} & \multicolumn{1}{l|}{1.200326} & 9.370801 \\ \hline
\end{tabular}
\end{table}

\begin{table}[H]
\centering
\caption{用MPI\_Type\_create\_struct聚合进程内变量后通信}
\label{表2}
\begin{tabular}{|c|lllll|}
\hline
\multirow{2}{*}{进程数} & \multicolumn{5}{c|}{矩阵规模}                                                                        \\ \cline{2-6} 
 & \multicolumn{1}{l|}{128} & \multicolumn{1}{l|}{256} & \multicolumn{1}{l|}{512} & \multicolumn{1}{l|}{1024} & 2048 \\ \hline
1                    & \multicolumn{1}{l|}{0.007141} & \multicolumn{1}{l|}{0.059567} & \multicolumn{1}{l|}{0.532740} & \multicolumn{1}{l|}{5.003699} & 52.291290 \\ \hline
2                    & \multicolumn{1}{l|}{0.003733} & \multicolumn{1}{l|}{0.029872} & \multicolumn{1}{l|}{0.280494} & \multicolumn{1}{l|}{2.591877} & 26.501993 \\ \hline
4                    & \multicolumn{1}{l|}{0.098602} & \multicolumn{1}{l|}{0.120185} & \multicolumn{1}{l|}{0.144042} & \multicolumn{1}{l|}{1.723110} & 15.240982 \\ \hline
8                    & \multicolumn{1}{l|}{0.027828} & \multicolumn{1}{l|}{0.057403} & \multicolumn{1}{l|}{0.231252} & \multicolumn{1}{l|}{1.130620} & 10.078613 \\ \hline
16                   & \multicolumn{1}{l|}{0.080558} & \multicolumn{1}{l|}{0.122373} & \multicolumn{1}{l|}{0.127154} & \multicolumn{1}{l|}{1.051537} & 8.649403 \\ \hline
\end{tabular}
\end{table}


\section{实验分析}
根据运行时间,分析程序并行性能及扩展性:

\subsection{MPI集合通信实现}
\begin{table}[H]
\centering
\caption{MPI集合通信实现的加速比}
\label{表3}
\begin{tabular}{|c|lllll|}
\hline
\multirow{2}{*}{进程数} & \multicolumn{5}{c|}{矩阵规模}                                                                        \\ \cline{2-6} 
 & \multicolumn{1}{l|}{128} & \multicolumn{1}{l|}{256} & \multicolumn{1}{l|}{512} & \multicolumn{1}{l|}{1024} & 2048 \\ \hline
1                    & \multicolumn{1}{l|}{1.00} & \multicolumn{1}{l|}{1.00} & \multicolumn{1}{l|}{1.00} & \multicolumn{1}{l|}{1.00} & 1.00 \\ \hline
2                    & \multicolumn{1}{l|}{2.11} & \multicolumn{1}{l|}{1.61} & \multicolumn{1}{l|}{1.95} & \multicolumn{1}{l|}{2.00} & 1.76 \\ \hline
4                    & \multicolumn{1}{l|}{3.61} & \multicolumn{1}{l|}{2.45} & \multicolumn{1}{l|}{3.10} & \multicolumn{1}{l|}{3.42} & 2.96 \\ \hline
8                    & \multicolumn{1}{l|}{0.26} & \multicolumn{1}{l|}{3.79} & \multicolumn{1}{l|}{3.53} & \multicolumn{1}{l|}{5.22} & 4.83 \\ \hline
16                   & \multicolumn{1}{l|}{0.18} & \multicolumn{1}{l|}{6.89} & \multicolumn{1}{l|}{5.12} & \multicolumn{1}{l|}{5.77} & 5.80 \\ \hline
\end{tabular}
\end{table}

\begin{itemize}
    \item 小规模矩阵(128~256):
    \begin{itemize}
        \item 进程数为2和4时,加速比随进程数增加而提高
        \item 进程数为8和16时,由于通信开销增加,加速比反而下降
        \item 256×256矩阵在16进程时达到最高加速比6.89
    \end{itemize}
    \item 中大规模矩阵(512~2048):
    \begin{itemize}
        \item 随着矩阵规模增大,加速比整体呈上升趋势
        \item 2048×2048矩阵在16进程时达到最高加速比5.80
        \item 进程数增加带来的性能提升逐渐趋于稳定
    \end{itemize}
\end{itemize}

\subsection{MPI\_Type\_create\_struct实现}
\begin{table}[H]
\centering
\caption{MPI\_Type\_create\_struct实现的加速比}
\label{表4}
\begin{tabular}{|c|lllll|}
\hline
\multirow{2}{*}{进程数} & \multicolumn{5}{c|}{矩阵规模}                                                                        \\ \cline{2-6} 
 & \multicolumn{1}{l|}{128} & \multicolumn{1}{l|}{256} & \multicolumn{1}{l|}{512} & \multicolumn{1}{l|}{1024} & 2048 \\ \hline
1                    & \multicolumn{1}{l|}{1.00} & \multicolumn{1}{l|}{1.00} & \multicolumn{1}{l|}{1.00} & \multicolumn{1}{l|}{1.00} & 1.00 \\ \hline
2                    & \multicolumn{1}{l|}{1.91} & \multicolumn{1}{l|}{1.99} & \multicolumn{1}{l|}{1.90} & \multicolumn{1}{l|}{1.93} & 1.97 \\ \hline
4                    & \multicolumn{1}{l|}{0.07} & \multicolumn{1}{l|}{0.50} & \multicolumn{1}{l|}{3.70} & \multicolumn{1}{l|}{2.90} & 3.43 \\ \hline
8                    & \multicolumn{1}{l|}{0.26} & \multicolumn{1}{l|}{1.04} & \multicolumn{1}{l|}{2.30} & \multicolumn{1}{l|}{4.42} & 5.19 \\ \hline
16                   & \multicolumn{1}{l|}{0.09} & \multicolumn{1}{l|}{0.49} & \multicolumn{1}{l|}{4.19} & \multicolumn{1}{l|}{4.76} & 6.05 \\ \hline
\end{tabular}
\end{table}

\begin{itemize}
    \item 小规模矩阵(128~256):
    \begin{itemize}
        \item 进程数为2时获得较好的加速比(1.91~1.99)
        \item 进程数超过2后,由于结构体通信开销,加速比显著下降
        \item 256×256矩阵在2进程时达到最高加速比1.99
    \end{itemize}
    \item 中大规模矩阵(512~2048):
    \begin{itemize}
        \item 随着矩阵规模增大,结构体通信的开销影响逐渐减小
        \item 512×512矩阵在16进程时达到最高加速比4.19
        \item 2048×2048矩阵在16进程时达到最高加速比6.05
    \end{itemize}
\end{itemize}

\subsection{扩展性分析}
\begin{itemize}
    \item MPI集合通信实现:
    \begin{itemize}
        \item 使用Bcast/Scatter/Gather进行数据分发和收集
        \item 在小规模矩阵下,扩展性能随进程数增加而下降
        \item 在大规模矩阵下表现出良好的扩展性
    \end{itemize}
    \item MPI\_Type\_create\_struct实现:
    \begin{itemize}
        \item 使用Bcast/Scatterv/Gather进行数据分发和收集
        \item 在小规模矩阵下,扩展性能同样随进程数增加而下降
        \item 在大规模矩阵下同样表现出良好的扩展性
    \end{itemize}
\end{itemize}

\end{document}